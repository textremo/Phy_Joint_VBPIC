% general packages
\usepackage{url}
\usepackage{graphicx}
\usepackage{mathtools}  % math tools
\usepackage{caption}    % support subfigure
\usepackage{subcaption} % support subfigure
\usepackage{placeins}   % support FloatBarrier
\usepackage{listings}   % support codes
\usepackage{color}      % support color
\usepackage{amsmath} 	% support multiple lines
\usepackage{amsfonts,amssymb} % support holo characters
\usepackage[10pt]{extsizes}
%\documentstyle[nips14submit_09,times,art10]{article} % For LaTeX 2.09

%-----------------------------------------------
% special packages

\usepackage[numbers,sort&compress]{natbib}		% 文献引用连号时[1,2,3]变成[1-3]
\usepackage{fancyhdr} 					% 加载fancyhdr宏包
\usepackage[hidelinks]{hyperref}        % 支持超链接(隐藏超链接效果)
\usepackage{multirow} 					% 多行表格
\usepackage{float} 						% 图片禁止浮动
\usepackage{geometry}
%\usepackage{CJKutf8} 					% 支持中文——方案1(不支持中文页眉而放弃)
\usepackage[UTF8]{ctex}  				% 支持中文——方案2
\usepackage{makecell} 					% 单单元格调整
\usepackage{longtable} 					% 表格——跨页面
\usepackage{tabularx} 					% 表格——自动调整列宽
\usepackage{array} 						% 表格——用于 \newcolumntype
\usepackage{makecell} 					% 表格——独立表格
\usepackage[table]{xcolor}  			% 加载xcolor宏包,启用表格颜色功能
\usepackage{pifont} 					% 圆圈1-20: \ding{172} -\ding{181} 
\usepackage{algorithm, algpseudocode} 	% 伪代码
\usepackage{underscore} 				% 下划线
\usepackage{ulem} 					% 文字划线等
\usepackage{svg} 								% 支持SVG图片
\usepackage{bm}									%
\usepackage{enumitem}							% item支持更多格式
\usepackage[numbers,sort&compress]{natbib}		% 文献引用连号时[1,2,3]变成[1-3]
\usepackage{breqn}								% 公式过长自动换行

%-----------------------------------------------
% page settings
\geometry{a4paper,left=2cm,right=2cm,top=2cm,bottom=2cm}
\renewcommand{\baselinestretch}{1.2}
\setlength{\parindent}{0pt} % 设置段落缩进为 0
% 定义页眉风格
\pagestyle{fancy} 
\fancyhead[L]{} % 设置左页眉
\fancyhead[C]{Qu's Joint Channle Estimation \& Symbol Detection} % 设置中页眉
\fancyhead[R]{} % 设置右页眉
\renewcommand{\headrulewidth}{0.4pt} % 设置页眉线宽度(可选)

%-----------------------------------------------
% set math notations
\DeclarePairedDelimiter{\norm}{\lVert}{\rVert}
% paths 
\graphicspath{{../../img/}}
% define color
\definecolor{darkred}{rgb}{0.6,0.0,0.0}
\definecolor{darkgreen}{rgb}{0,0.50,0}
\definecolor{lightblue}{rgb}{0.0,0.42,0.91}
\definecolor{orange}{rgb}{0.99,0.48,0.13}
\definecolor{grass}{rgb}{0.18,0.80,0.18}
\definecolor{pink}{rgb}{0.97,0.15,0.45}
\definecolor{lightgreen}{RGB}{220, 255, 220}  % 浅绿色
\definecolor{lightred}{RGB}{255, 220, 220}    % 浅红色

%-----------------------------------------------
% 其他自定义设置
% 覆盖 breqn 的编号设置
\renewcommand{\theequation}{\arabic{equation}}

% 定义不同宽度的列类型
%\newcolumntype{TBX}{>{\hsize=1\hsize\centering\arraybackslash}X}
%\newcolumntype{MyCol}{>{\hsize=1.5\hsize\centering\arraybackslash}X}
%\newcolumntype{TBL}{>{\hsize=2\hsize\arraybackslash}X}
% 允许跨页
\allowdisplaybreaks[4]
% 设置标题编号深度,使其出现在目录中(如果需要)
\setcounter{secnumdepth}{4}
\setcounter{tocdepth}{4} % 如果你也希望它出现在目录中
% 伪代码
% 伪代码——跨行
\makeatletter
\newenvironment{breakablealgorithm}
  {% \begin{breakablealgorithm}
   \begin{center}
     \refstepcounter{algorithm}% New algorithm
     \hrule height.8pt depth0pt \kern2pt% \@fs@pre for \@fs@ruled
     \renewcommand{\caption}[2][\relax]{% Make a new \caption
       {\raggedright\textbf{\ALG@name~\thealgorithm} ##2\par}%
       \ifx\relax##1\relax % #1 is \relax
         \addcontentsline{loa}{algorithm}{\protect\numberline{\thealgorithm}##2}%
       \else % #1 is not \relax
         \addcontentsline{loa}{algorithm}{\protect\numberline{\thealgorithm}##1}%
       \fi
       \kern2pt\hrule\kern2pt
     }
  }{% \end{breakablealgorithm}
     \kern2pt\hrule\relax% \@fs@post for \@fs@ruled
   \end{center}
  }
\makeatother
% 伪代码——注释
\renewcommand{\algorithmiccomment}[1]{\hfill // #1}


%% 自定义字符
\newcommand{\T}{^{\mathrm{T}}} % 转置
\newcommand{\HT}{^{\mathrm{H}}} % 共轭转置
\newcommand{\diag}{{\sf diag}} % 生成对角矩阵
\newcommand{\off}{{\sf off}} % 对角元素置0
\newcommand{\vect}{{\sf vec}} % 向量化